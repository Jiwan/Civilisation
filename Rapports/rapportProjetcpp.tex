%%%%%%%%%%%%%%%%%%%%%%%%%%%%%%%%%%%%%%%%%
% University Assignment Title Page 
% LaTeX Template
%
% This template has been downloaded from:
% http://www.latextemplates.com
%
% Original author:
% WikiBooks (http://en.wikibooks.org/wiki/LaTeX/Title_Creation)
% 
% Instructions for using this template:
% This title page is presently capable of being compiled as is. This is not 
% useful for including it in another document. To do this, you have two options: 
%
% 1) Copy/paste everything between \begin{document} and \end{document} 
% starting at \begin{titlepage} and paste this into another LaTeX file where you 
% want your title page.
% OR
% 2) Remove everything outside the \begin{titlepage} and \end{titlepage} and 
% move this file to the same directory as the LaTeX file you wish to add it to. 
% Then add \input{./title_page_1.tex} to your LaTeX file where you want your
% title page.
%
%%%%%%%%%%%%%%%%%%%%%%%%%%%%%%%%%%%%%%%%%

%----------------------------------------------------------------------------------------
%	PACKAGES AND OTHER DOCUMENT CONFIGURATIONS
%----------------------------------------------------------------------------------------

\documentclass[12pt]{article}

\begin{document}

\begin{titlepage}

\newcommand{\HRule}{\rule{\linewidth}{0.5mm}} % Defines a new command for the horizontal lines, change thickness here

\center % Center everything on the page
 
%----------------------------------------------------------------------------------------
%	HEADING SECTIONS
%----------------------------------------------------------------------------------------

\textsc{\LARGE INSA de Rennes}\\[1.5cm] % Name of your university/college
\textsc{\Large Programmation Orientée Objet}\\[0.5cm] % Major heading such as course name
\textsc{\large Projet C++}\\[0.5cm] % Minor heading such as course title

%----------------------------------------------------------------------------------------
%	TITLE SECTION
%----------------------------------------------------------------------------------------

\HRule \\[0.4cm]
{ \huge \bfseries Rapport de conception}\\[0.4cm] % Title of your document
\HRule \\[1.5cm]
 
%----------------------------------------------------------------------------------------
%	AUTHOR SECTION
%----------------------------------------------------------------------------------------

\begin{minipage}{0.4\textwidth}
\begin{flushleft} \large
\emph{Author:}\\
Jean \textsc{Guegant}\\ % Your name
Richard \textsc{Lagrange} % Your name
\end{flushleft}
\end{minipage}
~
\begin{minipage}{0.4\textwidth}
\begin{flushright} \large
\emph{Supervisor:} \\
Alexandru \textsc{Costan} % Supervisor's Name
\end{flushright}
\end{minipage}\\[4cm]

% If you don't want a supervisor, uncomment the two lines below and remove the section above
%\Large \emph{Author:}\\
%John \textsc{Smith}\\[3cm] % Your name

%----------------------------------------------------------------------------------------
%	DATE SECTION
%----------------------------------------------------------------------------------------

{\large \today}\\[3cm] % Date, change the \today to a set date if you want to be precise

%----------------------------------------------------------------------------------------
%	LOGO SECTION
%----------------------------------------------------------------------------------------

%\includegraphics{Logo}\\[1cm] % Include a department/university logo - this will require the graphicx package
 
%----------------------------------------------------------------------------------------

\vfill % Fill the rest of the page with whitespace

\end{titlepage}

\tableofcontent

\section{Introduction}

	Lors de la modélisation de notre version de \textit{Civilization}, nous avons voulu la rendre la plus flexible et extensible possible.
Nous voulions pouvoir facilement ajouter la gestion des jeux en réseau, pouvoir étendre et créer ses propres classes, etc.
Heureusement, ceci est réalisable grâce aux \textit{design patterns} (qui nous sont aussi imposés).
Nous avons essayé d'aller plus loin que les 5 patrons de conceptions imposé, afin de réaliser notre objectif d'un jeu pouvant évoluer.
A chaque fois, nous avons justifié l'utilisation du patron de conception.
	Bonne lecture.

\section{Cas d'utilisation}
	\subsection{Règles du jeu}
	Dans ce diagramme, nous avons illustré les intéractions entre les conceptes du jeu (les règles du jeu), les différentes unités intervenant dans une partie et les joueurs qui interagissent avec ces derniers.
	Les joueurs sont ceux qui créent des villes, créent des unités, bougent et font combattre ces unités et capturent des villes.
	Sur ce diagramme figurent les principales intéractions entre les joueurs.
	Nous avons fait un diagramme \textit{Développement d'une civilisation} pour illustrer les actions qu'un joueur doit entreprendre lors d'une partie, ainsi que les dépendances de ces actions.
	
	
	\subsection{Cycle de vie d'une partie}


	
\section{Diagrammes de Classe}
	\subsection{Civilization}
	\subsection{Player}
	\subsection{View}
	\subsection{World}
	
	
\section{Diagramme d'états-transitions}

\section{Diagrammes d'intégration}






































\end{document}
