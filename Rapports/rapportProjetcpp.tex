%%%%%%%%%%%%%%%%%%%%%%%%%%%%%%%%%%%%%%%%%
% University Assignment Title Page 
% LaTeX Template
%
% This template has been downloaded from:
% http://www.latextemplates.com
%
% Original author:
% WikiBooks (http://en.wikibooks.org/wiki/LaTeX/Title_Creation)
% 
% Instructions for using this template:
% This title page is presently capable of being compiled as is. This is not 
% useful for including it in another document. To do this, you have two options: 
%
% 1) Copy/paste everything between \begin{document} and \end{document} 
% starting at \begin{titlepage} and paste this into another LaTeX file where you 
% want your title page.
% OR
% 2) Remove everything outside the \begin{titlepage} and \end{titlepage} and 
% move this file to the same directory as the LaTeX file you wish to add it to. 
% Then add \input{./title_page_1.tex} to your LaTeX file where you want your
% title page.
%
%%%%%%%%%%%%%%%%%%%%%%%%%%%%%%%%%%%%%%%%%

%----------------------------------------------------------------------------------------
%	PACKAGES AND OTHER DOCUMENT CONFIGURATIONS
%----------------------------------------------------------------------------------------

\documentclass[12pt]{article}

\begin{document}

\begin{titlepage}

\newcommand{\HRule}{\rule{\linewidth}{0.5mm}} % Defines a new command for the horizontal lines, change thickness here

\center % Center everything on the page
 
%----------------------------------------------------------------------------------------
%	HEADING SECTIONS
%----------------------------------------------------------------------------------------

\textsc{\LARGE INSA de Rennes}\\[1.5cm] % Name of your university/college
\textsc{\Large Programmation Orientée Objet}\\[0.5cm] % Major heading such as course name
\textsc{\large Projet C++}\\[0.5cm] % Minor heading such as course title

%----------------------------------------------------------------------------------------
%	TITLE SECTION
%----------------------------------------------------------------------------------------

\HRule \\[0.4cm]
{ \huge \bfseries Rapport de conception}\\[0.4cm] % Title of your document
\HRule \\[1.5cm]
 
%----------------------------------------------------------------------------------------
%	AUTHOR SECTION
%----------------------------------------------------------------------------------------

\begin{minipage}{0.4\textwidth}
\begin{flushleft} \large
\emph{Author:}\\
Jean \textsc{Guegant}\\ % Your name
Richard \textsc{Lagrange} % Your name
\end{flushleft}
\end{minipage}
~
\begin{minipage}{0.4\textwidth}
\begin{flushright} \large
\emph{Supervisor:} \\
Alexandru \textsc{Costan} % Supervisor's Name
\end{flushright}
\end{minipage}\\[4cm]

% If you don't want a supervisor, uncomment the two lines below and remove the section above
%\Large \emph{Author:}\\
%John \textsc{Smith}\\[3cm] % Your name

%----------------------------------------------------------------------------------------
%	DATE SECTION
%----------------------------------------------------------------------------------------

{\large \today}\\[3cm] % Date, change the \today to a set date if you want to be precise

%----------------------------------------------------------------------------------------
%	LOGO SECTION
%----------------------------------------------------------------------------------------

%\includegraphics{Logo}\\[1cm] % Include a department/university logo - this will require the graphicx package
 
%----------------------------------------------------------------------------------------

\vfill % Fill the rest of the page with whitespace

\end{titlepage}

\tableofcontent

\section{Introduction}

	Lors de la modélisation de notre version de \textit{Civilization}, nous avons voulu la rendre la plus flexible et extensible possible.
Nous voulions pouvoir facilement ajouter la gestion des jeux en réseau, pouvoir étendre et créer ses propres classes, etc.
Heureusement, ceci est réalisable grâce aux \textit{design patterns} (qui nous sont aussi imposés).
Nous avons essayé d'aller plus loin que les 5 patrons de conceptions imposé, afin de réaliser notre objectif d'un jeu pouvant évoluer.
A chaque fois, nous avons justifié l'utilisation du patron de conception.
	Bonne lecture.

\section{Cas d'utilisation}
	\subsection{Règles du jeu}
		Dans ce diagramme, nous avons illustré les intéractions entre les conceptes du jeu et les joueurs. (Schéma 1)
	Les joueurs sont ceux qui créent des villes, créent des unités, bougent et font combattre ces unités et capturent des villes.
	Les dépendances entre les entités du jeu apparaissent bien.
	(Nous avons fait un diagramme \textit{Développement d'une civilisation} pour illustrer les actions qu'un joueur doit entreprendre lors d'une partie,
	ainsi que les dépendances de ces actions. JAI ENVIE D'EN METTRE QU'UN)
	
	\subsection{Cycle de vie d'une partie}
	Ce diagramme illustre toutes les phases du logiciel, avec les choix potentiels des joueurs.(Schéma 2)
	
\section{Diagrammes de Classe}
	\subsection{Civilization}
		\subsubsection{Namespace Civilization}
			Nous avons utilisé le design pattern Prototype pour pouvoir avoir le plus possible de civilizations différentes.
		En effet, au départ nous étions partis sur le patron de conception Fabrique Abstraite, ce qui aurait donné le Schéma 3.
			Après réflexion, nous voulions pouvoir étendre le nombre de civilizations pour pouvoir en ajouter autant que voulu.
		Le joueur pourrait importer sa civilisation créée (sous forme d'un fichier XML par exemple). 
		Donc pour implémenter cette solution, le choix du prototype est plus judicieux pour la Fabrique Concrète. (Schéma 4)
		Il permet d'éviter un nombre de sous classe important, qu'on aurait obtenu avec un patron Fabrique Abstraite classique (pour chaque civilisation, implémenter une fabrique concrète).
			Donc en somme, nous avons une Fabrique Abstraite, qui possède une Fabrique Concrète qui implémente le patron de conception Prototype.
			Cette Fabrique Concrète a donc un constructeur qui a comme rôle de cloner les interfaces des unités spécifiques d'une civilisation (Directeur, étudiant, etc.),
		pour en fabriquer d'autres, avec des caractéristiques uniques.
			Ces instances particulières des unités peuvent provenir de 2 endroits :
				1) Soit il les a crée lui même (via une interface utilisateur, qui créera un fichier XML contenant le prototype sérialisé)
				2) Soit ce sont des unités prédéfinies que l'on aura créées nous mêmes. (Il faudra alors désérialiser le prototype, qui est stocké sous la forme d'un fichier XMLcomme pour la 1ère méthode.)
		
		\subsubsection{Namespace Unit}
			Une interface commune IUnit est pratique si jamais on décide de faire une unité qui a une attaque qui dépend de plusieurs facteurs.
		(Comme attaque de base + attaque de l'épée + attaque du compagnon). Une interface commune permet de réaliser ceci facilement.
			La classe Unit est \textit{ISerializable} (comme expliqué auparavant) et \textit{IDrawable} (pour avoir les méthodes d'affichage).
		On a aussi des interfaces spécifiques pour les types d'unités spéciales (IDirector, IStudent, ITeacher), ce qui nous donne une certaine flexibilité.
			Si jamais on décide de comment est géré le bonus des directeurs de département par exemple, on pourra facilement changer ceci avec cette interface.
		
		\subsubsection{Namespace City}
			De même que pour IUnit, nous avons pensé à une interface ICity (qui pourrait se revéler pratique si nous voulons étendre un autre concept à des villes de type différent).
		Par exmple, nous pourrions imaginer une ville dont \textit{TotalAvailableFood} dépend aussi de son commerce avec une autre ville par exemple, nous pouvons changer la manière dont est implémenté cette méthode.
			Ensuite, nous avons utilisé le pattern Stratégie afin d'implémenter des algorithmes différents d'extension de la ville (\textit{"Le choix de cette case, sera réalisé par un algorithme que vous développerez."})
		Ainsi, on pourra au départ implémenter un algorithme simple et bête qui choisira une case aléatoirement.
			Puis dans un deuxième temps, choisir d'implémenter un algorithme plus intelligent, qui choisira les case en fonction de leur contenu exploitable (minerai, nourriture).
		De plus, vu qu'une seule instance de cet algorithme est utile à la fois, nous utiliseront le patron de conception Singleton.
			
		
	\subsection{Player}
	\subsection{View}
	\subsection{World}
		\subsubsection{Namespace World}
			\subsubsection{Namespace Square}
				Dans le namespace Square, nous avons crée une classe pour représenter une case du "damier" de notre jeu.
			Ces cases peuvent être de natures différentes (Montagne, Plaine, Désert) et avoir des ressources spéciales (Fer, Fruit).
			Grâce à ces natures et resources spéciales, on peut avoir de nombreux types de cases différents.
			Une implémentation idéale de ce cas serait le patron de conception Décorateur.
			En effet, grâce au décorateur, on peut alors "décorer" les cases Montagne, Plaine ou Désert avec les ressources spéciales Fer et Fruit.
			Il est aussi facile d'ajouter une nature ou une ressource spéciales, sans tout de même faire exploser le nombre de classes.
			
			\subsubsection{Namespace Map}
				L'interface IMapCreate permet d'obliger les classes de type Map qui l'implémentent d'avoir une méthode CreateMap qui est spéciale en fonction du randomizer qui est passé en paramètre.
			En fait, il s'agit du patron de conception Stratégie appliquée à une interface.
			Le lien entre l'interface IMapCreate et ISquareRandomizer n'est pas aussi fort qu'une aggrégation, mais doit quand même avoir un randomizer passé en paramètre.
			Il est alors facile d'avoir des stratégies différentes de création de cartes, si on veut des cartes avec plus de mer par exemple, ou plus de montagne.
			
	
\section{Diagramme d'états-transitions}

\section{Diagrammes d'intégration}






































\end{document}
